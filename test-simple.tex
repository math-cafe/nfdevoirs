% \documentclass[correction]{nfdevoirs}
\documentclass[correctionfin,theme=violet]{nfdevoirs}
% \documentclass{nfdevoirs}
% \documentclass[theme=nb]{nfdevoirs}

\title{EVA : Test simple}

\begin{document}

\begin{devoir}{
  calculatrice=autorisée,
  classe={1MATHS 2},
  date={29 septembre 2025},
  easteregg={hypothénuse},
  citation={Les mathématiques sont l'alphabet avec lequel Dieu a écrit l'univers.},
  auteur={Galilée},
  duree={1H30}
}

\begin{partie}{Équations du second degré}
  
  \begin{exercice}{Application directe}
    
    Résolvez les équations suivantes :
    
    \begin{question}{3}{0}
      $3x^2 - 2x + 1 = 0$
    \end{question}
    
    \begin{correction}
      Calculons le discriminant : $\Delta = b^2 - 4ac = 4 - 12 = -8 < 0$
      
      L'équation n'a pas de solution réelle.
    \end{correction}
    
    \begin{question}{3}{0}
      $x^2 - 6x + 9 = 0$
    \end{question}
    
    \begin{correction}
      $\Delta = 36 - 36 = 0$
      
      Une solution double : $x_0 = \frac{6}{2} = 3$
    \end{correction}
    
  \end{exercice}
  
  \begin{exercice}{Exercice Bonus}
    
    Cet exercice est entièrement bonus.
    
    \begin{question}{0}{5}
      Montrez que $(2\sqrt{3} + 1)^2 = 13 + 4\sqrt{3}$
    \end{question}
    
    \begin{correction}
      $(2\sqrt{3} + 1)^2 = 4 \times 3 + 2 \times 2\sqrt{3} + 1 = 12 + 4\sqrt{3} + 1 = 13 + 4\sqrt{3}$
    \end{correction}
    
  \end{exercice}
  
\end{partie}

\end{devoir}

\end{document}